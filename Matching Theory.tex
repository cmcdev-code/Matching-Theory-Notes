\documentclass[12pt]{article}
 
\usepackage[margin=1in]{geometry} 
\usepackage{amsmath,amsthm,amssymb,graphicx,mathtools,tikz,hyperref}
\usetikzlibrary{positioning}
\newcommand{\n}{\mathbb{N}}
\newcommand{\z}{\mathbb{Z}}
\newcommand{\q}{\mathbb{Q}}
\newcommand{\cx}{\mathbb{C}}
\newcommand{\real}{\mathbb{R}}
\newcommand{\field}{\mathbb{F}}
\newcommand{\ita}[1]{\textit{#1}}
\newcommand{\com}[2]{#1\backslash#2}
\newcommand{\oneton}{\{1,2,3,...,n\}}
\newcommand\idea[1]{\begin{gather*}#1\end{gather*}}
\newcommand\ef{\ita{f} }
\newcommand\eff{\ita{f}}
\newcommand\proofs[1]{\begin{proof}#1\end{proof}}
\newcommand\inv[1]{#1^{-1}}
\newcommand\setb[1]{\{#1\}}
\newcommand\en{\ita{n }}
\newcommand{\vbrack}[1]{\langle #1\rangle}

\newenvironment{theorem}[2][Theorem]{\begin{trivlist}
\item[\hskip \labelsep {\bfseries #1}\hskip \labelsep {\bfseries #2.}]}{\end{trivlist}}
\newenvironment{lemma}[2][Lemma]{\begin{trivlist}
\item[\hskip \labelsep {\bfseries #1}\hskip \labelsep {\bfseries #2.}]}{\end{trivlist}}
\newenvironment{exercise}[2][Exercise]{\begin{trivlist}
\item[\hskip \labelsep {\bfseries #1}\hskip \labelsep {\bfseries #2.}]}{\end{trivlist}}
\newenvironment{reflection}[2][Reflection]{\begin{trivlist}
\item[\hskip \labelsep {\bfseries #1}\hskip \labelsep {\bfseries #2.}]}{\end{trivlist}}
\newenvironment{proposition}[2][Proposition]{\begin{trivlist}
\item[\hskip \labelsep {\bfseries #1}\hskip \labelsep {\bfseries #2.}]}{\end{trivlist}}
\newenvironment{corollary}[2][Corollary]{\begin{trivlist}
\item[\hskip \labelsep {\bfseries #1}\hskip \labelsep {\bfseries #2.}]}{\end{trivlist}}
 \hypersetup{
 colorlinks,
 linkcolor=blue
 }
\begin{document}
\date{}


\title{Matching Theory Notes}
\author{Collin McDevitt}

\maketitle

\begin{lemma}{1.0.1}
    For any graph, $G$, $\alpha (G)+\tau(G)=|V(G)|$.
\end{lemma}

\begin{proof}
    Let $G$ be an arbitrary graph and let $M$ be an arbitrary point cover where $|M|=\tau(G)$.
    Then $V(G)-M$ is an independent set of points. This is true because if $V(G)-M$ was not an independent set of points then
    there are at least two points $u,v \in V(G)-M$ where $u$ and $v$ are adjacent. If they are adjacent the line incident with $u$ and $v$ is not covered by $M$ which is a contradiction.
    As $V(G)-M$ is an independent set of points we get the inequality $$\alpha(G) \geq |V(G)-M|= |V(G)|-\tau (G)$$

    Now assume that $L$ is an arbitrary independent set of points where $|L|=\alpha(G)$. Then $V(G)-L$ is a point cover of $G$. If $V(G)-L$ is not a point cover of $G$ then there exists a line $l$ such that $l$ is not covered by $V(G)-L$. However this would imply that $l$ is incident with two points in $L$ this is a contradiction on the fact $L$ is independent points hence $V(G)-L$ is a point cover of $G$. As $V(G)-L$ is a point cover of $G$ we get
    the inequality $$\tau(G) \leq |V(G)-L|=|V(G)|-\alpha(G)$$

    Combining both inequalities we get $\alpha(G) \geq |V(G)|-\tau(G)$ and $\tau(G) \leq |V(G)|-\alpha(G)$. This implies that $$\alpha(G)+\tau(G)=|V(G)|$$.
\end{proof}

\begin{lemma}{1.0.2}
    For any graph $G$ with no isolated points, $v(G)+\rho(G)=|V(G)|$.

\end{lemma}

\begin{proof}
    Let $G$ be an arbitrary graph with no isolated points and let $C$ be a line cover of $G$ where $|C|=\rho(G)$. Let $\langle C \rangle$ be the graph formed from lines the set of lines $C$ and the set of points $V(C)$. We have that $\langle C \rangle$ is a union of stars. This is because if $\langle C \rangle$ was not a union of stars. Then there would be two points in the graph that that are adjacent two one or more point in $\langle C \rangle$. If they are adjacent to a single point then removing either of the lines incident would create a smaller minimal line cover. If they are adjacent to two or more points then removing any of the incident lines would create a smaller minimal line cover.
    Now if we let $n$ be the number of components in $\langle C \rangle$ we get that $n=|V(G)|-\rho(G)$ which arises because in each star there is one more point than there are lines. If we take one line from each star we get a matching hence $$v(G)\geq |V(G)| - \rho(G)$$

    Now let $M$ be an arbitrary matching of $G$ where $|M|=v(G)$, and $U$ be the set of points that are not covered by $M$. We get that $U$ is an independent set of points because if $U$ was not an independent set of points then there would be two points $u,v \in U$ such that $u$ and $v$ are adjacent. If $u$ and $v$ are adjacent then there is a line incident with $u$ and $v$ that is not covered by $M$ which is a contradiction. As $U$ is an independent set of points and $G$ has no isolated points we get that $|U| = |V(G)|-2v(G)$. Let $S$ be a line covering of $U$ we get that $M\cup S$ is a line covering. Then we get the inequality $$\rho (G)\leq |M \cup S|= v(G)+|V(G)|-2v(G)=|V(G)|-v(G)$$ Combining both inequalities we get $$v(G)+\rho(G)=|V(G)|$$
\end{proof}


\begin{exercise}{1.0.3}
    \text{\\}

    \begin{enumerate}
        \item A minimal line cover is minimum if and only if it contains a maximum matching.
        \item A maximal matching is maximum if and only if it is contained in a minimum line cover.
    \end{enumerate}
\end{exercise}

\begin{proof}{(1)}

    Let $L$ be a line cover where $|L|=\rho(G)$. As $L$ is a minimum line cover
    we have that it forms the lines from a union of stars on $G$. We have that the number of stars is $v(G)$ which comes from selecting one matching from each star. Hence $L$ contains a maximum matching.

    Let $L$ be a minimal line cover that contains a maximum matching. Then as $L$ is a union of stars we get a maximum matching by selecting a line from each star. Hence $|L|=|V(G)|-v(G)=\rho(G)$ hence it is a minimum line cover.
\end{proof}

\begin{proof}{(2)}
    Let $M$ be a maximum matching. Let $L$ be an line cover obtained by taking an arbitrary line covering of the edges incident with $V(G)-V(M)$ and $M$. We have that $|L|=|V(G)|-2v(G)+v(G)=\rho(G)$ hence $L$ contains a minimum line cover.

    Let $M$ be a maximal matching that is contained in a minimum line cover $L$.
    Then we have that the matchings in $M$ are formed from individual lines of the stars formed by $L$. Hence $|M|=V(G)-\rho(G)=v(G)$ hence $M$ is a maximum matching.
\end{proof}


\begin{exercise}
    {1.0.4}
    For any graph $G$, $v(G)\leq \tau(G)$
\end{exercise}

\begin{proof}

    Let $G$ be an arbitrary graph and let $M$ be a maximum matching in $G$. As $M$ is a maximum matching we get that it is contained in a minimum line cover $L$. As $L$ is a minimum line cover we have that it is the union of stars where each matching in $M$ comes from a single line in each star. We get that the number of stars is $\tau(G)$ hence $v(G)\leq \tau(G)$
\end{proof}
\begin{theorem}
    {1.1.1}
    König's Minimax Theorem:
    If $G$ is bipartite, then $v(G)=\tau(G)$
\end{theorem}

\begin{proof}
    Remove lines from $G$ as long as possible while keeping $\tau(G)$ the same. Denote the resulting minimal graph by $G^\prime$. Hence $\tau(G^\prime)=\tau(G)$, but for any line $l\in E(G)$ we have $\tau(G^\prime -l)<\tau(G)$. Suppose that $G^\prime$ does not consist of independent lines. Then there are two lines $l_1, l_2 \in E(G^\prime)$ that are both incident with a point $a \in V(G^\prime)$. As $G^\prime$ is minimal we have that there exists a point cover of $G^\prime-x$ we denote this point cover by $S_x$ where $|S_x|=\tau(G^\prime)-1$. We also have the point cover $S_y$ which is covering $G^\prime -y$ with $|S_x|=|S_y|$.
    Form the induced subgraph $G^{\prime \prime}$ of $G^\prime, \; G^{\prime \prime}=G^\prime[\{a\}\cup S_x\cup S_y \cap \overline{S_x \cap S_y}]$ and let $t=|S_x \cap S_y|$. Then we have $|V(G^{\prime \prime})|=2(\tau(G^\prime)-1 -t)+1$. As $G^{\prime \prime}$ is a subgraph of a bipartite graph we have that it is bipartite. Hence we have a set $T$ which is a point cover of $G^{\prime \prime}$ where $|T|=\tau(G^\prime)-1 -t$.
    We claim that the set of points $T^\prime = T \cup (S_x \cap S_y)$ covers $G^\prime$. Let $z$ be a line in $G^\prime$ if $z\not = x $ or $z\not = y$ then we have that $z$ is covered by $S_x$ and $S_y$ hence it is covered by $S_x \cap S_y$ or it connects $S_x-S_y$ to $S_y-S_x$. If $z=x$ or $z=y$ then we have it is covered by $T$. So $\tau(G^\prime)\leq |T^\prime|= |T\cup (S_x\cap S_y)|=|T|+|S_x\cap S_y|\leq\tau(G^\prime)-1-t+t=\tau(G^\prime)-1$ a contradiction. Hence $G^\prime$ is an independent set of lines. Hence $$\tau(G)=\tau(G^\prime)=v(G^\prime)\leq v(G)$$

\end{proof}


\begin{theorem}
    {1.1.2}

    In any (possibly infinite) bipartite graph there exists a matching $M$ and a point cover $P$ such that every line in $M$ contains exactly one point in $P$ and every point in $P$ is contained in exactly one line of $M$.
\end{theorem}

\begin{proof}
    Let $G$ be an arbitrary bipartite graph and let $L$ be a minimum point cover of $G$. Create the new graph $G^{\prime}$ by removing lines from $G$ as long as $\tau(G^\prime)=\tau(G)$. We have that $G^\prime$ consists of an independent set of lines from each of the points in the point cover $\tau(G^\prime)$ select one line. As $G^{\prime}$ consists of an independent set of lines we have a maximum matching by König's Minimax Theorem which satisfies the implication.
\end{proof}

If $X$ is any set of points in a graph $G$, we denote the set of all points which are adjacent to at least one point of $X$ as $\Gamma(X)$.

\begin{theorem}
    {1.1.3}
    (P. Hall's Theorem). Let $G=(A,B)$ be a bipartite graph. Then $G$ has a matching of $A$ into $B$ if and only if $|\Gamma(X)|\geq |X|$ for all $X \subseteq A$.
\end{theorem}

\begin{proof}

    Let $G=(A,B)$ be a bipartite graph and suppose that there is a matching $A$ to $B$ and suppose that for some subgraph $X \subseteq A$ we have $|\Gamma(X)|<|X|$. If $X$ is any subgraph of $A$ and there exists a matching from $A$ to $B$ we have that there exists a matching from $X$ to $B$ as well. As $|\Gamma(X)|<|X|$ we have that there is at least $1$ point in $X$ that is not adjacent to any point in $B$ hence it can not be a complete matching.

    Suppose $|\Gamma(X)|\geq |X|$ for all $X\subseteq A$. Then using mathematical induction on $|A|$. If $|A|=0$ then we would have a complete matching. If $|A|=1$ then by our assumption we have that $|\Gamma(A)|\geq |A|$ hence there is a matching.


    Case 1: \\

    Suppose that for all $X \subset A,\; X\not = \emptyset, |X|< |\Gamma(X)|$ holds. Let $a\in A,b \in B$ be two adjacent points. Let $G^{\prime}= G - a - b$ and let $X$ be any subset of $A-a$. If $X = \emptyset$, then $|X|=|\Gamma_{G^\prime}(X)|$ so assume $X \not = \emptyset$. Since $X\not = A,\; |X|< |\Gamma_{G^\prime}(X)|$ based on the assumption then we have $|\Gamma_{G^\prime}(X)|\geq |\Gamma_{G}(X)|-1\geq|X|$. Therefore by the induction hypothesis, there is a matching $M^\prime$ of $G^\prime$ which covers all points of $A-a$. But then $M=M^\prime \cup \{ab\}$ matches $A$ and $B$ as desired.

    Case 2:\\
    Suppose there is a set $A^\prime \subset A, \; A^\prime \not = \emptyset$ with $|\Gamma_G(A^\prime)|=|A^\prime|$. We proceed to split $G$ into two smaller subgraphs by letting $G_1$ be the subgraph induced by $A^\prime \cup \Gamma(A^\prime)$ and $G_2=G-A^\prime-\Gamma(A^\prime)$. Suppose $X\subseteq A^\prime$. Then $\Gamma_G(X)\subseteq \Gamma_G(A^\prime)$, so $\Gamma_{G_1}(X)=\Gamma_G(X)$ and hence $|\Gamma_{G_1}(X)|=|\Gamma_G(X)|\geq |X|$. Now in $G_2$ assume $X\subseteq A- A^\prime$. Then $\Gamma_G(X\cup A^\prime)=\Gamma_{G_2}(X)\cup \Gamma_G(A^\prime)$ and therefore $|\Gamma_{G_2}(X)|=|\Gamma_G(X\cup A^\prime)|-|\Gamma_{G}(A^\prime)|\geq |X\cup A^\prime|-|\Gamma_G(A^\prime)|=|X\cup A^\prime|-|A^\prime|=|X|$.

    By applying the induction hypothesis to both $G_1$ and $G_2$, we see that there must exist matchings $M_1$ of $A^\prime$ into $\Gamma_G(A^\prime)$ and $M_2$ of $A-A^\prime$ into $B-\Gamma_G(A^\prime)$. The union of $M=M_1 \cup M_2$ is the desired matching.


\end{proof}

\begin{corollary}
    {1.1.4}(The Marriage Theorem). A bipartite graph $G=(A,B)$ has a prefect matching if and only if $|A|=|B|$ and for each $X\subseteq A,\; |X|\leq |\Gamma(X)|$.
\end{corollary}
\begin{proof}
    Let $G$ be a bipartite graph with $G=(A,B)$ and assume that $G$ has a prefect matching $M$. Then we have $|A|=|B|$ otherwise it would not be a prefect matching. By Hall's theorem we have for each $X\subseteq A,\; |X|\leq |\Gamma(X)|$.


    Assume $G$ is a bipartite graph $G=(A,B)$ where $|A|=|B|$ and for each $X \subseteq A,\; |X|\leq |\Gamma(X)|$. Then by Hall's Theorem we have that there is a matching $M$ of $A$ into $B$. However as we have $|A|=|B|$ we get that this is a prefect matching.
\end{proof}


\begin{corollary}
    {1.1.7}
    If $G$ is bipartite, $\rho(G)=\alpha(G)$.
\end{corollary}
\begin{proof}
    Suppose $G$ is a bipartite graph then by König's Minimax Theorem we have $v(G)=\tau(G)$ by the Gallai Identities we have $$v(G)=|V(G)|-\rho(G)$$ and $$\tau(G)=|V(G)|-\alpha(G)$$ combining with König's Minimax Theorem we get $$|V(G)|-\rho(G)=|V(G)|-\alpha(G)$$
    $$\rho(G)=\alpha(G)$$
\end{proof}

\begin{theorem}
    {1.2.1}
    Let $M$ be a matching in a graph $G$. Then $M$ is a maximum matching if and only if there exists no augmenting path in $G$ relative to $M$.
\end{theorem}



\begin{proof}

    Let $G$ be a graph and $M$ be a maximum matching in $G$ and assume that there exists a augmenting path $P$ in $G$ relative to $M$. Then consider the matching $M^\prime = (M \setminus M \cap P) \cup (P \setminus M)$ this would imply $|M|< |M^\prime|$ which is a clear contradiction.


    Let $G$ be a graph and $M$ be a matching that is not maximum and suppose there are no augmenting paths in $G$ relative to $M$. As $M$ is not a maximum matching there exists a larger matching $M$. Create the graph $G^\prime =G[M\setminus M^{\prime} \cup M^{\prime}\setminus M]$ as this creates an alternating path and there are at least two points in $V(G^\prime)$ that are not in $V(M)$ we have an augmenting path.
\end{proof}


\begin{lemma}
    {1.2.2}
    If $G=(A,B)$ is a bipartite graph and $M$ is a matching on $G$ where
    $A_1\subseteq A,B_1\subseteq B$ are the sets of exposed points where $F$ is the maximal forest in $G$ with the properties:
    \begin{itemize}
        \item For each point $b$ of $F$ has degree $2$ and of the incident lines of $b$ is in $M$.
        \item Each component of $F$ contains a point in $A_1$.
    \end{itemize}
    ,then $M$ is a maximum matching if and only if no point of $B_1$ is adjacent to any point of $F$.

\end{lemma}

\begin{proof}
    Assume $M$ is a maximum matching and there exists a point $b$ of $B_1$ that is adjacent to a point in $F$. Then we have an $M-augmenting$ path which by Theorem 1.2.1 implies $M$ is not a maximum matching.

    Suppose that no point of $B_1$ is adjacent to any point of $F$.
    We have $M$ covers $A\setminus V(F)\cup (B\cap V(F))$ which is immediate as $M$ covers $A$ and $B$. We have that no line of $M$ joins a point from $A\setminus (F)$ to a point of $B\cap V(F)$. As if there was then we would have both points would have to be in $V(F)$ however that would contradict $A\setminus V(F)$. Now as $M$ covers $A\setminus V(F)\cup (B\cap V(F))$ and  each line of $M$ contains exactly one single point of $A\setminus V(F)\cup (B\cap V(F))$. We have $|A\setminus V(F)\cup (B\cap V(F))|=|M|$.

    Now to show that $A\setminus V(F)\cup (B\cap V(F))$ is a point cover. Suppose that it is not and there is a line $ab$ that is not covered by $A\setminus V(F)\cup (B\cap V(F))$ where $a\in A$ and $b\in B$ then we have $a\in V(F)$ and $b\not \in V(F)$. By the hypothesis we have $b\not \in B_1$ therefore the matching covers $b$ by a line $a^\prime b$. We have $a\not = a^\prime$

\end{proof}



\end{document}